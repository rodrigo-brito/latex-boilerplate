\documentclass{article}
\usepackage[tmargin=0.5in]{geometry}
\usepackage[brazil,portuges]{babel}
\usepackage[utf8]{inputenc}
\usepackage{graphicx}
\usepackage{amsmath,amssymb}
\usepackage{url}
\usepackage[ruled,lined]{algorithm2e}
\usepackage{booktabs}
\graphicspath{{fig/}} % Location of the graphics files
\usepackage[round, authoryear]{natbib}

\begin{document}

\title{Latex Boilerplate}
\author{Rodrigo Brito}

\maketitle

\begin{abstract}
Não sendo obrigatório, mas você pode incluir um resumo ao seu trabalho. Para remover, basta excluir este bloco de comandos;
\end{abstract}

\section{Introdução}
Escreva aqui suas informações introdutórias, podendo dividi-las em subseções entre outros diversos tipos de elementos como mostrado nas seções seguintes.

\subsection{Equações e fórmulas matemáticas}

A Equação \ref{simple_equation} mostra um simples exemplo de modelo matemático assinalado com \textit{label}, também podendo ser utilizados em linha, assim: $\alpha = \sqrt{ \beta }$.

\begin{equation}
    \label{simple_equation}
    \alpha = \sqrt{ \beta }
\end{equation}

\subsection{Subseção}
Write your subsection text here.

\begin{figure}[!htp]
    \centering
    \includegraphics[width=0.5\linewidth]{fig/image.png}
    \caption{Simulation Results}
    \label{figura}
\end{figure}

\subsection{Tabelas e Quadros}

A Tabela \ref{tabela} mostra um exemplo simples de organização, uma maneira simplificada de gerar a estrutura é através do site \url{http://www.tablesgenerator.com/}.
\begin{table}[h]
	\centering
	\begin{tabular}{lll}
		\hline
		& Descida  & Simulated Annealing \\ \hline
		FO        & 50       & 45,7                \\
		Tempo (s) & 3,817595 & 73,43069            \\ \hline
	\end{tabular}	
	\caption{Exemplo de tabela}
	\label{tabela}
\end{table}

\subsection{Pseudocódigos}

No Algoritmo \ref{algoritmo1} é mostrado um exemplo de pseudocódigo em \LaTeX.

\begin{algorithm}[H]
	\SetAlgoLined
	\Entrada{$S,\eta, U$} 
	\Saida{Número esperado de nodos atingidos}
	\Inicio{
		$\sigma(S) = 0$ \\
		\Se{$u = S$}{
			$\sigma(S)\leftarrow \sigma(S)+\textsc{Backtrack}(u,\eta,W,U)$\\
		}
	}
	\Retorna{$\sigma(S)$}
	\label{algoritmo1}
	\caption{\textsc{Esperança}}
\end{algorithm}

\subsection{Citações}
Para citar referência em linha \cite{knuth:84}, e para fim de linha \citep{knuth:84}. Você também pode citar figuras, assim como a Figura \ref{figura}, que é referenciada pelo seu \textit{label}.

\section{Conclusões}
Escreva suas conclusões aqui

\bibliographystyle{citacoes-abnt}
\bibliography{referencias}

\end{document}
